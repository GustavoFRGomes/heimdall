\chapter{Introduction}
\label{chap:intro}

The purpose of this first chapter is to present the subject of the project as
well as the reasons behind choosing it. This chapter will also present the scope
for the project, its objectives and the organization of this document.

\section{Context}
\label{sec:context} % CADA SECÇÃO DEVE TER UM LABEL
Nowadays we live in an increasingly interconnected world where more and more
devices share the Internet. Mostly this landscape consists of devices that are
an interface between the user and all the information present in the Internet.
Lately there has been a shift and the number of devices that mostly interact
between themselves, these devices comprise the backbone of the Internet of
Things, IoT for short.

It is estimated that by 2020 there will be aproximatedly 30 billions devices
connected to the Internet, most of which won't be directly controled by human
beings and will range from smart applicances to surveillance web cameras or
automobilies, etc.
These devices will collect data from the real world and extract knowledge
from all that data in order to help improve the users quality of life.

The emergence of smart appliances has already begun as well as the success of
newer automobiles that are Internet enabled given them not just software updates
over the air but also extracting data that will help develop new features. These
are all devices that operate, mostly, by themselves, like for example the smart
appliances that order groceries whenever their sensors identify any product
which is missing from the users daily life, or the surveillance cameras that
allow the user to check in on it's house from anywhere on the planet.

Since these devices won't interact directly with the users, there isn't a need
for big amounts of processing power, and because these devices need to be
spreaded across the entire world to be effective they need to be  cheap to
manufacture also culminating in a lack of computational resources that exceed
the devices initial plan.
This of course makes these devices somewhat vunerable security wise because they
are normally designed with one task in mind and security isn't a priority for
most manufacturers because it would increase costs on both the software
development but also at the hardware level because sometimes there needs to be
an increase in the devices resources.

% Os acrónimos devem ser definidos recorrendo ao pacote (\emph{package}) \texttt{acronym}, usando os comandos \texttt{\textbackslash acro}, \texttt{\textbackslash ac}, \texttt{\textbackslash acp}, etc. E.g., \emph{The subject of this report is network protocols, namely \ac{TCP}.  \ac{TCP} is studied for several aspects of performance.}


\section{Motivation}
\label{sec:mot}
As mentioned in the previous section most of the devices that will comprise the
Internet of Things (IoT) may not have the most basic security features demanded
today. Most don't make use of encryption when communicating with the another
devices which makes them target for traffic interception, for example.
This leads to an increasing amount of stories describing how malicious attackers
intercepted and stole information from these devices and in some cases these
Internet enables "things" can become the point of entry for the manlicious
purpotraters to invade our domestic networks.

The challenge here is to make it possible for normal people to block or at least
monitor to some extent the devices inside their domestic network, mostly these
IoT devices in order to shield the network from malicious actors. This is also
important because mot manufacturer keep shipping IoT devices that lack even the
most basic security features, despite the recent uptake by some technological
blogs and websites.

\section{Objectives}
% Goals would be better instead of objectives?
\label{sec:obj}
This project will try to provide the user with a way to prevent or block all of
the connection he or she deems unsafe, by providing the user with a configurable
firewall at the gateway level it is possible for them to flag or block unsecure
connections that might happen between their IoT devices and the Internet.

The main objectives of this project are to develop a system able to replace the
network gateway, using a Raspberry Pi. The functionalities that this system
should have are listed below:
\begin{enumerate}
	\item Restriction of outgoing connections from the local network;
	\item Blocking such connections within the gateway;
	\item Using the Tor network to make all connections anonymous;
	\item Email notifications whenever a flagged connection is atempted.
\end{enumerate}

\section{Distribution of Content}
\label{sec:organ}
% !POR EXEMPLO!
% De modo a refletir o trabalho que foi feito, este documento encontra-se estruturado da seguinte forma:
In order to describe to some extent all the work behind this finished product
the report will have the following structure:
\begin{enumerate}
	% 1
	\item The first chapter -- \textbf{Introduction} -- talks about the context
		in wich the project will strive, it will also present the goals and the
		motives behind it's inception.
	% 2
	\item The second chapter -- \textbf{Related Works and Technology} --
		presents similar solutions to this particular project as well as the
		differentiationg factors between these solutions and this project at
		hand. It will also describe the technology used to conceive the system.
	% 3
	\item The third chapter -- \textbf{System Architecture} -- consists of a
		small requirement analysis, description of the architecture and also the
		tools used.
	% 4
	\item The forth chapter -- \textbf{System Description and Implementation} --
		is a description of all of its components as well as the implementation
		decisions involved in the creation of the subsystems.
	% 5
	\item The fifth chapter -- \textbf{Prototype and Testing} -- talks about the
		prototype as well as the setup in which the system was tested. It will
		also talk about how the system should be setup in real life, its
		installation and correct configuratin and usage.
	% 6
	\item The sixth chapter -- \textbf{Conclusions and Future Work} -- will
		consist in a description of the main conclusions extracted from the
		development of this system as well as what is going to be developed
		further in this system.
\end{enumerate}

%s
% \section{Algumas Dicas -- Retirar da Versão Final}
% ALGUMAS DICAS
% Os relatórios de projeto são individuais e preparados em \LaTeX, seguindo o formato disponível na página da unidade curricular. Deve ser prestada especial atenção aos seguintes pontos:
% \begin{enumerate}
%   \item O relatório deve ter um capítulo Introdução e Conclusões e Trabalho Futuro (ou só Conclusões);
%   \item A última secção do primeiro capítulo deve descrever suscintamente a organização do documento;
%   \item O relatório pode ser escrito em Língua Portuguesa ou Inglesa;
%   \item Todas as imagens ou tabelas devem ter legendas e ser referidas no texto (usando comando \texttt{\textbackslash ref\{\}}).
% \end{enumerate}
