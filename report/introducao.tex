\chapter{Introduction}
\label{chap:intro}

\section{Contextualization}
\label{sec:context} % CADA SECÇÃO DEVE TER UM LABEL
% CADA FIGURA DEVE TER UM LABEL
% CADA TABELA DEVE TER UM LABEL
We live in an information era, one that is driven by the use of more and more
devices connected to the Internet. These devices form the Internet of Things
(IoT). They could be a laptop, smartphone or even the new smart appliances in
our kitchen are connected to the Internet, which try to aid our everyday lives
by telling us that we ran out of milk or that there is strange movement around
house house even when you are at work.

% Os acrónimos devem ser definidos recorrendo ao pacote (\emph{package}) \texttt{acronym}, usando os comandos \texttt{\textbackslash acro}, \texttt{\textbackslash ac}, \texttt{\textbackslash acp}, etc. E.g., \emph{The subject of this report is network protocols, namely \ac{TCP}.  \ac{TCP} is studied for several aspects of performance.}


\section{Motivation}
\label{sec:mot}
Devices that are connected to the Internet are normally very limited when it
comes down to the computational resources at their disposal. This lack of
computational power leads to more relaxed security because those same security
measures that prevent intrusions or spying need some resources on their own.

As it can be expected big manufacturers of these devices won't care if their
product could be accessed through the Internet even by someone who's not their
client. This is very common unfortunately and there are more and more cases
reaching the public's eyes.

The challenge behind this project is to devise a way for any user be shielded
from the manufacturers negligent decisions and also notify the network user to
potential usage of unsecure protocols by their smart devices.

\section{Objectives}
% Goals would be better instead of objectives?
\label{sec:obj}
The project aims to protect every user's domestic network from potential
intrusions and avoid any privacy violations as well. The user could be
abstracted by the complexities of network configuration but at the same time
define exclusion or notifications for the usage of unsecure protocol.

The main objectives of this project is to develop a system able to replace the
network gateway, using a Raspberry Pi. The functionalities that the system
shoult have are, as follows:
\begin{enumerate}
	\item Restriction of outgoing connections from the local network;
	\item Blocking such connections within the gateway;
	\item Using the Tor network to make all connections anonymous;
	\item Email notifications whenever a flagged connection is atempted.
\end{enumerate}

\section{Distribution of Content}
\label{sec:organ}
% !POR EXEMPLO!
% De modo a refletir o trabalho que foi feito, este documento encontra-se estruturado da seguinte forma:
In order to describe to some extent all the work behind this finished product
the report will have the following structure:
\begin{enumerate}
	% 1
	\item The first chapter -- \textbf{Introduction} -- talks about the context
		in wich the project will strive, it will also present the goals and the
		motives behind it's inception.
	% 2
	\item The second chapter -- \textbf{Related Works and Technology} --
		presents similar solutions to this particular project as well as the
		differentiationg factors between these solutions and this project at
		hand. It will also describe the technology used to conceive the system.
	% 3
	\item The third chapter -- \textbf{System Architecture} -- consists of a
		small requirement analysis, description of the architecture and also the
		tools used.
	% 4
	\item The forth chapter -- \textbf{System Description and Implementation} --
		is a description of all of its components as well as the implementation
		decisions involved in the creation of the subsystems.
	% 5
	\item The fifth chapter -- \textbf{Prototype and Testing} -- talks about the
		prototype as well as the setup in which the system was tested. It will
		also talk about how the system should be setup in real life, its
		installation and correct configuratin and usage.
	% 6
	\item The sixth chapter -- \textbf{Conclusions and Future Work} -- will
		consist in a description of the main conclusions extracted from the
		development of this system as well as what is going to be developed
		further in this system.
\end{enumerate}

%s
% \section{Algumas Dicas -- Retirar da Versão Final}
% ALGUMAS DICAS
% Os relatórios de projeto são individuais e preparados em \LaTeX, seguindo o formato disponível na página da unidade curricular. Deve ser prestada especial atenção aos seguintes pontos:
% \begin{enumerate}
%   \item O relatório deve ter um capítulo Introdução e Conclusões e Trabalho Futuro (ou só Conclusões);
%   \item A última secção do primeiro capítulo deve descrever suscintamente a organização do documento;
%   \item O relatório pode ser escrito em Língua Portuguesa ou Inglesa;
%   \item Todas as imagens ou tabelas devem ter legendas e ser referidas no texto (usando comando \texttt{\textbackslash ref\{\}}).
% \end{enumerate}
