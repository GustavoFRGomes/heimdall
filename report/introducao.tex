\chapter{Introduction}
\label{chap:intro}

\section{Contextualization}
\label{sec:context} % CADA SECÇÃO DEVE TER UM LABEL
% CADA FIGURA DEVE TER UM LABEL
% CADA TABELA DEVE TER UM LABEL
We live in an information era, one that is driven by the use of more and more
devices connected to the Internet. These devices form the Internet of Things
(IoT). They could be a laptop, smartphone or even the new smart appliances in
our kitchen are connected to the Internet, which try to aid our everyday lives
by telling us that we ran out of milk or that there is strange movement around
house house even when you are at work.

Os acrónimos devem ser definidos recorrendo ao pacote (\emph{package}) \texttt{acronym}, usando os comandos \texttt{\textbackslash acro}, \texttt{\textbackslash ac}, \texttt{\textbackslash acp}, etc. E.g., \emph{The subject of this report is network protocols, namely \ac{TCP}.  \ac{TCP} is studied for several aspects of performance.}


\section{Motivation}
\label{sec:mot}
Devices that are connected to the Internet are normally very limited when it
comes down to the computational resources at their disposal. This lack of
computational power leads to more relaxed security because those same security
measures that prevent intrusions or spying need some resources on their own.

As it can be expected big manufacturers of these devices won't care if their
product could be accessed through the Internet even by someone who's not their
client. This is very common unfortunately and there are more and more cases
reaching the public's eyes.

The challenge behind this project is to devise a way for any user be shielded
from the manufacturers negligent decisions and also notify the network user to
potential usage of unsecure protocols by their smart devices.

\section{Objectives}
% Goals would be better instead of objectives?
\label{sec:obj}
The project aims to protect every user's domestic network from potential
intrusions and avoid any privacy violations as well. The user could be
abstracted by the complexities of network configuration but at the same time
define exclusion or notifications for the usage of unsecure protocol.

\section{Organização do Documento}
\label{sec:organ}
% !POR EXEMPLO!
De modo a refletir o trabalho que foi feito, este documento encontra-se estruturado da seguinte forma:
\begin{enumerate}
\item O primeiro capítulo -- \textbf{Introdução} -- apresenta o projeto, a motivação para a sua escolha, o enquadramento para o mesmo, os seus objetivos e a respetiva organização do documento.
\item O segundo capítulo -- \textbf{Tecnologias Utilizadas} -- descreve os conceitos mais importantes no âmbito deste projeto, bem como as tecnologias utilizadas durante do desenvolvimento da aplicação.
\item ...
\end{enumerate}

%s
\section{Algumas Dicas -- Retirar da Versão Final}
% ALGUMAS DICAS
Os relatórios de projeto são individuais e preparados em \LaTeX, seguindo o formato disponível na página da unidade curricular. Deve ser prestada especial atenção aos seguintes pontos:
\begin{enumerate}
  \item O relatório deve ter um capítulo Introdução e Conclusões e Trabalho Futuro (ou só Conclusões);
  \item A última secção do primeiro capítulo deve descrever suscintamente a organização do documento;
  \item O relatório pode ser escrito em Língua Portuguesa ou Inglesa;
  \item Todas as imagens ou tabelas devem ter legendas e ser referidas no texto (usando comando \texttt{\textbackslash ref\{\}}).
\end{enumerate}
