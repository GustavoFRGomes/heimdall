\chapter{Related Works and Technology}
% Os titulos dados aos capítulos são meros exemplos. Cada relatório deve adequar-se ao projeto desenvolvido.
\label{chap:related-material}

\section{Introduction}
\label{chap2:sec:intro}
% Cada capítulo \underline{intermédio} deve começar com uma breve introdução onde é explicado com um pouco mais de detalhe qual é o tema deste capítulo, e como é que se encontra organizado (i.e., o que é que cada secção seguinte discute).
In this chapter there will be presented some related works, with some similar
functionalities as the project at hand. There will also be a section dedicated
to the description of the underlying techonolgy that is going to be used in the
development of this project.

\section{Similar Solutions}
\label{chap2:sec:similar-sol}
There are similar projects but not with the same set of functionalities as the
project at hand. The similar solutions are as follows:
\begin{enumerate}
	\item Nagios
	\item Domestic Routers
	\item Wireshark
	\item Snort
\end{enumerate}

Now we begin the description of the products listed above and also pointing out
the main differences between these solutions and the project described
throughout this report.

\subsection{Nagios}
\label{chap2:sec:similar-sol:sub:nagios}
One very similar project is the Nagios software, and it consists of a network
infrastructure monitoring system. Developed mostly for comercial and industrial
environments. There is a special version made for the Raspberry Pi, called
NagiosPi, that enables the user to check all of the connected devices and
monitor their traffic they generate. The NagiosPi system has the following
interfaces:
\begin{enumerate}
	\item Nagios;
	\item Nconf;
	\item NagVis;
	\item PHPMyAdmin;
	\item RaspControl;
\end{enumerate}

The NagiosPi system permits that the netwrok administrator logs all of the
information it might find pertinent as well as monitor all of the traffic
circulating inside the network. Using these interfaces the users can monitor
specific ports and hosts, it could visualize the packets and some statistics.

One of the biggest differences to the project being described in this report is
that it is more focused towards the domestic environments, because Nagios has a
complex interface using the web browser and not the mobile application offered
by this project. One more difference is that Nagios exclusively handles
monitoring, but the system being developed also has firewall capabilities, which
means it could block and also notification capabilities for flagged protocols
and packets.

\subsection{Domestic Routers}
\label{chap2:sec:similar-sol:sub:routers}
Another similar solution that could be referenced is the domestic router which
people have in their homes. This router actually comes with a firewall already
minimally configured preventing that any connection which isn't started by a
device inside the domestic network is unable to access or even probe the devices
which are inside the domestic network.

As said before this is a simple firewall but make a lot of difference because it
prevents malicious attackers from getting information about the network itself
and any of the devices connected to it, as well as prevents the exploitation of
these devices becasue they need that device to access a resource on the Internet
in order to become infected. This might sound useless but the reason wy people
get infected by a computer virus is because they access a website or resource on
the Internet which spreads malware, so it is clear that the user inside the
domestic network needs to access the infectious resource or download it in order
to get infected by it, in other words, it needs to bring the infection to
himself and not be always vulnerable.

The main difference between the firewall system of domestic routers and the
system described in this project is that the domestic router filters incoming
traffic, packets and connections coming from the Internet to the domestic
network. In the case of this system the filtering is done on the outgoins
packets, so they complement each other, preventing the user from getting malware
and also making it possible to prevent any communication between the virus and
any other resource on the Internet.
Another difference is that our system can be easily configurable using the
mobile application as well as notify the user of flagged behaviors or
communications, which the home router can't do.
% \begin{lstlisting}[caption=Trecho de código usado no projeto.]
% #include <stdio.h>

% int main(){
%   int i = 0;
%   for(i = 0; i < 100; i++)
%     printf("%d\n",i);
% }
% \end{lstlisting}

\section{Raspberry Pi}
\label{chap2:sec:rasp}
The Raspberry Pi is a small computer that comprises all of the vital component
that make it a computer, with a CPU, memory and input and output capabilities.
This tiny computer was choosen due to its affordable price, less then 45€. For
that price the resources available are not extremly performant but for the
development of this gateway it is enough due to the low-level nature of the data
being accessed.

This micro-computer will have installed the Operating System and also the
project itself which comprises a web service, a database, a firewall and a
notification scrip for emails.

%talk about the specs of the Raspberry Pi 2?
Since one of the goals of this system is to develop a configurable gateway for
domestic networks, it is vital that the device that runs such gateway would be
inexpensive and the Raspberry Pi is a perfect solution because of it's good
price to power ratio.

\section{Android}
\label{chap2:sec:android}
For the interaction between the user and system running on the Raspberry Pi an
Android application will be used. The Android OS is one of the major mobile
operating systems used and it is mostly due to the fact that it can run on
devices with low computational resources so it is the perfect platform for the
development of the companion app for this system.

This mobile operating system operates using applications written in Java which
is also one of the most popular programming languages in the world, the Android
OS is also equipped with a vast set of tools and libraries that can be leveraged
to create a web service consumer which makes it a perfect fit for the purposes
of this project.

% make a point that Android has inexpensive devices like smartphones and tables.

\section{Conclusions}
\label{chap2:sec:concs}
This chapter talked about some of the similar products and also the main
differentiating characteristics between these solutions and our project, like
the Nagios monitoring system, or the firewall of domestic routers.
In this chapter there was also a description of the main technologies being
leveraged in this project, the Raspberry Pi inexpensive computer and also the
Android mobile operating system.
