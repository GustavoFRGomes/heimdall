\chapter{State of the Art}
% OU \chapter{Trabalhos Relacionados}
% OU \chapter{Engenharia de Software}
% OU \chapter{Tecnologias e Ferramentas Utilizadas}
\label{chap:estado-da-arte}

\section{Introdução}
\label{chap2:sec:intro}
Nowadays there has been a number of cases that work reveal the insecurity of
most devices on the Internet of Things
% put some references here maybe!!
but how do regular people protect themselves from the perils their devices
might bring to their household? There isn't a solution to protect their own
network from anyone who could leverage their "smart" appliances to get inside
it and perform malicious actions, like traffic monitoring, sensitive
information extraction or even make your devices part of a botnet.

The focus of this project is on the traffic that willbe sent from devices onto
the Internet, the outgoing traffic because it represents a point of failure.
If the outgoing traffic isn't secure then anyone could study our daily life,
form the information sent from any "smart" appliances an attacker could study
a potential victim and spy on their lives. At this present time there are a
number of cameras that are connected to the Intenret, thus being part of the
Internet of Things, that are reachable by anyone on the Internet and that
could be viewed by anyone who as a minimum skill level for michieves.

\section{Secções Intermediarias}
\label{chap2:sec:...}

\section{Conclusões}
\label{chap2:sec:concs}
Cada capítulo \underline{intermédio} deve referir o que demais importante se conclui desta parte do trabalho, de modo a fornecer a motivação para o capítulo ou passos seguintes.
