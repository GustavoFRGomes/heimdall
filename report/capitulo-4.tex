\chapter{System Description and Implementation}
% Os titulos dados aos capítulos são meros exemplos. Cada relatório deve adequar-se ao projeto desenvolvido.
\label{chap:sys-desc}

\section{Introduction}
\label{chap4:sec:intro}
% Cada capítulo \underline{intermédio} deve começar com uma breve introdução onde é explicado com um pouco mais de detalhe qual é o tema deste capítulo, e como é que se encontra organizado (i.e., o que é que cada secção seguinte discute).
This chapter will describe the four subsystem that were developed and that
compose the entire project. As said previously the entire system consists of
four subsystems that interact with one another using the database, the intention
is to make the system behave asynchronously and this will help in the
development or upgrade of any of the components since they are not linked they
can be exchanged for others which could have better performance or efficiency.

\section{Web Service Subsystem}
\label{chap4:sec:web-sys}
% divide it in Flask subsection and Android App subsection
The Web Service was done using the Flask micro-framework implemented in Python.
As already said in the previous chapter this framework allows for the use of
third-party extensions as the ones used to develop the web service API. These
extensions are:
\begin{enumerate}
	\item Flask-RESTful
	\item Flask-SQLAlchemy
\end{enumerate}

This system also can be divided into the Web Server and the Android Companion
application. These are made using a different technology stack, Python's Flask
and the Android Application, and because of this the communication between these
two systems need to deploy the same communication protocol and architecture.
This is accomplished by using the REST architecture. This is the

% \begin{lstlisting}[caption=Trecho de código usado no projeto.]

\section{Database Subsystem}
\label{chap4:sec:db-sys}
The database is the centerpiece of the entire system. It uses the SQLite
database management system and it is mostly interacted with using a Python
module named SQLAlchemy.

Using the \emph{SQLAlchemy} module we could declare classes that will represent
our tables and their attributes within the database on a Python script. This
way there is also an abstraction for the SQL queries, where the developer can
call methods to add, delete, update and filter through the data that is within
the database. This is accomplished, as said previously, by declaring classes to
represent the models of the data, and declaring the data types for each
attribute and some properties that they might have, e.g. declaring an attribute
as unique within a table.

% insert and explain at least one of the models

After modelling the data we only need to create a session for the database, and
all of the data inside the database will be accessible with the call of methods
from the session instance. These methods could be insert, delete, update and
query.

% insert and explian at least one query and one insertion

% The database is the central component of our system and it is the most
% important system, because without it nothing else can run.

The database is organized with the following schema:

%insert an image of the database's schema here

\section{Email Notification Subsystem}
\label{chap4:sec:email-sys}

\section{Firewall Handler Subsystem}
\label{chap4:sec:firewall-sys}
The Firewall Subsytem consists of the firewall itself, which will be
accomplished using the \emph{iptables} system, present in most Linux
distributions as the default firewall system, and there is also a firewall
handling system that is responsible for acquiring the rules, defined by the
user, from the database and deploying them on the firewall system.

\subsection{iptables}
\label{chap4:sec:firewall-sys:sub:iptables}
The iptables is an application program that allows for the configuration of the
Linux kernel firewall, which uses the Netfilter framework. This application,
provided in most Linux distributions, is an interface that enables users to
set up, maintain and inspect the tables of IP packet filtering rules within the
Linux kernel.

It is a widely used and very flexible system used by system administrators that
need to enforce certain security policies on a single system or network. There
are two versions of this software, one for IPv4 and another one for IPv6, the
latter being named iptables6.

\subsection{python-iptables}
\label{chap4:sec:firewall-sys:sub:iptc}
The \emph{python-iptables} is a package that makes the interface between
iptables and Python scripts. By using this package we could create the firewall
handling class that will interact with the Linux kernel firewall and employ the
rules using the Python language.
This Python package makes the process of deploying rules to the firewall easier
as well as the addition of new rule chains for the filtering within the tables.

The Firewall class does just that, querying all of the rules inside the
database, looking for optional source and destination IP addresses for any given
rule and implementing the pretended actions.
The main cycle of the class is the \emph{run()} method that will perform the
following task within an infinite loop:

\begin{enumerate}
	\item Get a list of rules from the DB;
	\item Flush the rules within the iptables ruleset;
	\item Insert the new rules;
	\item Wait for the refresh time to pass and start again.
\end{enumerate}

Some optimizations are also present, for example there is a match made by the
system in order to check if the ruleset from the DB is the same of the previous
iteration, or disabling the autocommit for each rule in order to deploy the
entire ruleset right after flushing instead of flushing the entire ruleset and
adding a rule at a time.

% Add the Firewall().run() method here!

\section{Conclusions}
\label{chap4:sec:concs}
This chapter described all of the subsystems that comprise the project. We
talked about the Web Service with the Flask server and also the Android app that
consumes it. We also described the Database system which is the central
component and all systems depend on it. We referred the Email Notification and
the details that need to be in place to make email notifications possible. At
last we described how the Firewall subsystem and how we accomplished the sense
of dynamic firewall configuring.
