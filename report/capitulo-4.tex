\chapter{System Description and Implementation}
% Os titulos dados aos capítulos são meros exemplos. Cada relatório deve adequar-se ao projeto desenvolvido.
\label{chap:sys-desc}

\section{Introduction}
\label{chap4:sec:intro}
% Cada capítulo \underline{intermédio} deve começar com uma breve introdução onde é explicado com um pouco mais de detalhe qual é o tema deste capítulo, e como é que se encontra organizado (i.e., o que é que cada secção seguinte discute).
This project contains a detailed description of each of the subsystem running on
the Raspberry Pi as well as the mobile application. As mentioned in the previous
chapter the system which will run on the Raspberry Pi is inspired by the concept
of Microservices meaning that each service provided by the gateway system is
independent from all of the other services, which means a detailed approach to
each service is necessary.

\section{Web Service Subsystem}
\label{chap4:sec:web-sys}

The Web Service is be responsible for all the communication between the user and
the Raspberry Pi components. The web service will be all implemented making use
of the \emph{Flask} framework allied with the \emph{flask-restful} extension
that enables the declaration of resources to be accessed through
\emph{\ac{HTTP}} requests.

In order to implement each resource we only need to create a class to represent
each \emhp{\ac{URI}}. These resources will handle the \emph{\ac{HTTP}} methods,
such as GET, POST, PUT and DELETE, according to their implementation providing
the following functionalities through the \emph{\ac{API}}:
\begin{enumerate}
	\item List the ruleset of the server;
	\item Add rules to the ruleset;
	\item Remove rules from the ruleset;
	\item Edit rules in the ruleset;
	\item Get the latest counter values for the entire ruleset;
	\item Get all of the values for a given rule.
\end{enumerate}

This Web Service implements a \emph{\ac{REST}} interface for the interaction
with external devices. The use of the \emph{\ac{REST}} \emph{\ac{API}} makes
interfacing with other devices more standard, opening the possibility to
implement other applications that access this \emph{\ac{API}}.
We now present the principles which need to be followed by the Web Service in
order to be considered a \emph{\ac{REST}} compliant web service:
\begin{enumerate}
	\item Client-Server architecture with a clear separation between the two.
	\item Stateless -- every request from the client must contain all the
		pertinent information, which includes authentication credentials, for
		example. The server should not store any session information about the
		client.
	\item Cache -- responses from the client might be chacheable or not by the
		server or intermidiate systems in order to maximing performance. These
		responses must be labeled by the server as cacheable or noncacheable.
	\item Uniform Interface -- the protocol used between the client and the
		server resources must be well defined and standardized, for exmaple the
		use of the HTTP protocol.
	\item Layered Interface -- other machines and systems might be added between
		the client and the server for reliability, scalability or performance
		reasons.
	\item Code-on-Demand -- optionally, clients can dowload and execute code
		from the server in their own context.
\end{enumerate}
(These principles were first described in Roy Fielding's Ph.D. dissertation.)

The Web Service will parse and respond with data in the \emph{\ac{JSON}} format
providing abstraction from the language being used for the client application.

The \emph{\ac{API}} for this RESTful web service can be accessed by appending
the following \emph{\ac{URI}} to the gateway's \emph{\ac{IP}} address:
\begin{enumerate}
	\item "rule/<int:id>" -- making it available for the client to check for a
		specific rule, as well as delete or edit it, using the GET, DELETE and
		PUT requests, respectively, from the \emph{\ac{HTTP}} protocol.
	\item "/rules" -- this \emph{\ac{URI}} will enable the client application to request for
		the full set of rules currently running on the server, by using the GET
		request. By making a request with the POST type the client application
		can append new rules to the rule set already in place, which will then
		be updated during the next cycle of the firewall subsystem.
	\item "/counter/<int:id>" -- URI that will only be accessed through GET
		request and provided with the id of the rule it will deliver the
		counters value, with number of packets and total number of bytes that
		were matched with that rule.
	\item "/counters" -- also only accessible through GET requests, which sends
		a list of the latest counters collected from the firewall service. These
		will be a list of rules identified by their ID, as well as the number of
		packets and bytes for that corresponding rule.
\end{enumerate}

% TALK ABOUT THE FIELDS OF THE flask-restful
In order to send all of the responses in a standard \emph{\ac{JSON}} format the
\emph{flask-restful} extension provides a formatter to convert the objects into
the \emph{\ac{JSON}} format intended. This is demonstrated in the following code
listing:

\begin{lstlisting}[caption=Example of the email JSON template.]
ip_fields = {
        'ip': fields.String,
        'ipv4': fields.Boolean,
        'src': fields.Boolean,
}

mac_fields = {
        'mac': fields.String,
}

rule_fields = {
        'id': fields.Integer,
        'port': fields.String,
        'protocol': fields.String,
        'action': fields.String,
        'ip': fields.Nested(ip_fields),
        'mac': fields.Nested(mac_fields),
}
\end{lstlisting}

This is then used by the \emph{marshal\_with} Python decorator in order to
convert the data object from the database into \emph{\ac{JSON}} format.

\begin{lstlisting}[caption=Example of the use of marshal\_with decorator.]
class RuleResource(Resource):
	# ...
	@marshal_with(rule_fields)
	def get(self, id):
		rule = session.query(Rule).filter(Rule.id == id).first()
		if not rule:
			abort(404, message="Can't find {0} in the DB.".format(id))
		return rule
	# ...
\end{lstlisting}

\section{Database Subsystem}
\label{chap4:sec:db-sys}

The Database is the center of the gateway system, where all of the data is
concentrated.
It is a \emph{SQLite} database mostly handled by the \emph{SQLAlchemy}, already
described in the Tools section. this subsystem doesn't have an executable
content, because it consists in the entity declaration for the data models
needed for this projet as well as two functions for the creation of the database
as well as sessions to it. All of this is concentrated in the \emph{models.py}
file within the root directory of the project.

The following image represents the database schema that was needed for the
system, including the relationships between entities:

% put the schema of the DB

As said previously the system makes use of the \emph{SQLAlchemy} module in order
abstract the programmer from creating the necessary \emph{\ac{SQL}} queries. The
entities of the database are introduced in the next list:
\begin{enumerate}
	\item User -- representing the user's credentials, username and password
		fields.
	\item Rule -- represents a rule for the firewall system to execute.
	\item Rule\_IP -- as the IP that the rule might have associated to it.
	\item Packet -- as a representation of a network packet, especially used for
		the notifications.
\end{enumerate}

These entities only need to be declared using plain Python objects as it can be
seen in the following code snippet:
% insert and explain at least one of the models
\begin{lstlisting}[caption=Example of the User entity declaration.]
class User(Base):
    __tablename__ = 'users'

    id = Column(Integer, primary_key=True)
    username = Column(String(255), unique=True)
    password = Column(String(512))
    timestamp = Column(String(100))

    def set_password(self, password):
        self.password = generate_password_hash(password)

    def check_password(self, password):
        return check_password_hash(self.password, password)

    def __repr__(self):
        return '<User {0}>'.format(self.username)
\end{lstlisting}

This was a small example of the declaration of the User entity for this project.
These objects will then be converted into \emph{\ac{SQL}} queries by the
\emph{SQLAlchemy} module, declaring the schema with teh appropriate data types
from the database being used. This type of conversion is importante when
treating relationships between entities as well as the use of types of data that
might need to be converted depending on the database system chosen by the
developer, for example the Boolean data type which is not present in most
database systems such as \emph{SQLite}

In order to access this data the developer should declare a session to the
database in order to handle all of the data. The next code snippet has the
session creator function, also present in the \emph{models.py} file:
\begin{lstlisting}[caption=Example of session creator function.]
def create_session(uri='sqlite:///./heimdall.sqlite3'):
    from sqlalchemy import create_engine
    from sqlalchemy.orm import scoped_session
    from sqlalchemy.orm import sessionmaker
    Session = sessionmaker(autocommit=True, autoflush=False, \
                           bind=create_engine(uri))
    return scoped_session(Session)
\end{lstlisting}

\section{Email Notification Subsystem}
\label{chap4:sec:email-sys}
The Email Notification Subsystem will act as a complement for the Android
application, this means that it will only send out notifications to the user if
he isn't home. In order to detect the user's presence, the Email Service make
use of the timstamp registration of each user in the database checking when was
the last time the user made requests to the web service. Afterwards it
calculates when how much time it passed and checks if it is greater then or
equal to the cycle period also specified by the user. If it is so, it will send
out a report of the activity on the network, and repeats all the sequence again.

It is up to the user to define the email cycle as well as to provide the email
address he or she wants to receive the reports and also the user's name. These
emails are all composed based on a \emph{\ac{JSON}} template located in the
\emph{email.json} file. The next code listing is an example of the template for
the email generation, the dynamic reports will be inserted in place of the
\emph{reason} value below:
\begin{lstlisting}[caption=Example of the email JSON template.]
{
    "email": {
        "users\_name": "Gustavo Gomes",
        "sign": "Sent From Heimdall Secure Gateway.",
        "goodbye": "Thank you for your attention,",
        "subject": "Notification From Heimdall Secure Gateway",
        "reason": "Nothing to report!",
        "email\_to": "gustavofrgomes@hotmail.com",
        "greeting": "Hello there Mr/Mrs"
    }
}
\end{lstlisting}

The email service will also be in charge of creating the small reports that go
in the reason field, which is the body of the email per se. This is all done
automatically by the \emph{mixRuleCounters(. ,.)} which takes as parameters the
rules and counters received from the database. The method will then iterate
through all of the rules which have \emph{NOTIFY} or \emph{NOTIFY\_BLOCK} as
actions and link them with the counters for those same rules.

The following image is an example of the email service whenever there is not
activity in the database:

% insert image of the email example

A problem found with this feature was that the email system globally has become
much more secure and it is impossible to send an email without an Email Provider
which will also imply that the user would need to purchase or get a domain name
in order to make this feature possible, the solution found for this problem was
to create an email account that will have all the credentials and be prepared to
be accessed by this subsystem, using the email address and the password of that
account and connecting via a TLS connection to the email provider, which in this
case is Google's Gmail. By doing this it is possible to send out emails to the
user's specified account from that account created for the system.

By making use of the \emph{email} module provided within Python's Standard
Library, it is possible to connect to an account on a SMTP server in order to
send any email, after the input of user's credentials such as the email address
and the password of that account. These credentials should be in a
\emph{\ac{JSON} format on the file named \emph{creds.json}.

\section{Firewall Interface Subsystem}
\label{chap4:sec:firewall-sys}
The Firewall Subsystem is the one component responsible for the management and
deployment of rules inside the firewall, iptables. The only concern for this
system is to deploy the rules from the database and the accounting for the
network traffic of those same rules. First we will talk about the
\emph{python-iptables}, or \emph{iptc} package, which enables the direct
interaction between this system and the firewall system, which was described in
a previous chapter, the \emph{iptables}. Then a description of the class that
manipulates and handles both the rules and their corresponding counters is
described as well as the processes involved with the data retrieval from the DB
and conversion.

\subsection{python-iptables}
\label{chap4:sec:firewall-sys:iptc}
The \emph{python-iptables} is a third party package that can be found in the
Python Package Index (PyPI), it provides a direct interface with the
\emph{iptables} firewall system. This is mostly due to the fact that it is a
Python wrapper for the \eph{libiptc} which, in turn is a C library that
provides a programming interface to the \emph{iptables}.

This module allows the access to any of the tables, chains and rules within the
\emph{iptables} system, as well as the addition of new rules to the system and
also the manipulation of existing ones. This means that it is possible to
dynnamically manipulate the rules inside the \emph{iptables} system in any way
the developer might require.
Some of the main functionalities that are provided by this module as well as
used in the context of this project are:
\begin{enumerate}
	\item Target declaration;
	\item Protocol definition;
	\item Port speficication;
	\item IP filtering, which includes IP ranges;
	\item MAC address filtering;
	% \item Pattern matching.
\end{enumerate}

The way these functionalities will be used is covered in the next chapter.

\subsection{Firewall Class}
\label{chap4:sec:firewall-sys:firewall-class}
There is a Firewall object that can be configured with the chains defined on
\emph{iptables}, this can be accomplished by setting defining the named optional
parameter when initializing a Firewall object, it also defaults to the 'FORWARD'
chain in the \emph{iptables} system which is the one that this subsystem needs
to interact with. It will always connect to the \emph{FILTER} table of
\emph{iptables} since this is the table that filters all of the traffic passing
through the machine, i.e. the Raspberry Pi.

Since the entire system is modular the Firewall class can be used independently
from the system itself. Of course it is still recommended that it should be
paired at least with the database since it has methods to query a database using
a \emph{SQLAlchemy} session.

The methods that are provided in this class are the following:
\begin{enumerate}
	\item addRule() -- method that handles the conversion of rules inside the
		database model object to \emph{python-iptables} rule objects.
	\item submitRule(.) -- handles the submission of the rule to the database,
		if the table from the database has the autocommit feature as False then
		it will simply add the rule but not commit it to the working firewall
		rule set.
	\item flushRules() -- responsible for flushing all of the rules from
		the chain.
	\item resetCounters() -- method that will reset the counters for all of the
		rules in the \emph{iptables}, as well as the data inside the database.
		This is used when the rules change because the counters need to start
		from zero and the database needs to be cleaned from the old rule set.
	\item ruleUpdate() -- main cycle for this subsystem, which gets the rules
		from the database, converts them from the models inside the database to
		\emph{iptables} compliant rules, resets the counters and adds the rules
		to the chain before ever commiting them to the firewall itself.
	\item createRule(.) -- creates an \emph{iptables} rule using a dictionary
		instead of an instance of the database models.
	\item counterUpdate() -- updates the counters inside the database, since the
		counters aren't dynamic per se, the table of the firewall needs to be
		refreshed in order to update the values of the counters. This will then
		publish the results in the database.
	\item counterDB(.) -- method that receives the results of the counters and
		connects them to each rule, afterwards sending it all to the database,
		by converting the data into a database model object.
	\item run(wait\_time) -- main method that will run both the ruleUpdate()
		method as well as the counterUpdate() to always keep the rules tracked
		and equal to the ones on the database. The \emph{wait\_time} param
		defaults to 30 and it is the periodicity of the cycle, in seconds.
\end{enumerate}

The Firewall Subsytem consists of the firewall itself, which will be
accomplished using the \emph{iptables} system, present in most Linux
distributions as the default firewall system, and there is also a firewall
handling system that is responsible for acquiring the rules, defined by the
user, from the database and deploying them on the firewall system.

\section{Android Companion App}
\label{chap4:sec:android}
In order to make it a complete system an Android mobile application was
conceived that mostly allows the user to interact with the Web Service's API.
The requests to the web service are made using the Volley as well as the
HTTPClient, since the Web Service only provides the API via HTTPS and with
authentication, the HTTPClient is required to handle the self-signed certificate
provided with the application as well as take care for the HTTPBasic
authentication headers.

As already said when describing the Web Service all of the data transmitted over
the network needs to be in the JSON format in order to provide a standard way of
communicating with all the client devices.

The aplication will irst ask the user to provide the authentication credentials
as well as the gateway's IP address.

% insert the authentication print

There are some classes defined to help structure all of the data and to allow
for some organization and manipulation of the data that can be received by the
Web Service. These classes are the Rule, Ip and Mac classes.

The activities that are provided by the application are presented in the
following list:
\begin{enumerate}
	\item AddRule -- Activity to append a rule at the end of the rule set on the
		server, uses the POST request;
	\item EditRule -- Used to edit a rule that is already implemented, making
		use of the PUT request method.
	\item DeleteRule -- Small activity that handles the DELETE request for a
		specific rule;
	\item ListRules -- Activity to list and access all of the rules that are in
		the system.
	\item Login -- First activity to be launched providing the user with a way
		to input both the credentials for authentication in the Web Service as
		well as the IP and Port number of the gateway.
	\item MainMenu -- This activity is just a simple menu to enable the user to
		interact with all of the functionalities proviced by the application.
\end{enumerate}

Almost all of the activities make use of the Volley package, described in
Chapter 2, for the communication between the Web Service and the application.
The application composes Hyper Text Tranfer Protocol (HTTP) requests to be sent
to the Web Server as well as JSONObject or JSONArray objects to send and parse
information received from the Web Service, running on the gateway device.

\section{Conclusions}
\label{chap4:sec:concs}
This chapter described all of the subsystems that comprise the project. We
talked about the Web Service with the Flask server and also the Android app that
consumes it. We also described the Database system which is the central
component and all systems depend on it. We referred the Email Notification and
the details that need to be in place to make email notifications possible. At
last we described how the Firewall subsystem and how we accomplished the sense
of dynamic firewall configuring.
