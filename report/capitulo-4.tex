\chapter{System Description and Implementation}
% Os titulos dados aos capítulos são meros exemplos. Cada relatório deve adequar-se ao projeto desenvolvido.
\label{chap:sys-desc}

\section{Introduction}
\label{chap4:sec:intro}
Cada capítulo \underline{intermédio} deve começar com uma breve introdução onde é explicado com um pouco mais de detalhe qual é o tema deste capítulo, e como é que se encontra organizado (i.e., o que é que cada secção seguinte discute).

\section{Web Service Subsystem}
\label{chap4:sec:web-sys}
The Web Service was done using the Flask micro-framework implemented in Python.
As already said in the previous chapter this framework allows for the use of
third-party extensions as the ones used to develop the web service API. These
extensions are:
\begin{enumerate}
	\item Flask-RESTful
	\item Flask-SQLAlchemy
\end{enumerate}

% \begin{lstlisting}[caption=Trecho de código usado no projeto.]

\section{Database Subsystem}
\label{chap4:sec:db-sys}
The database is the centerpiece of the entire system. It uses the SQLite
database management system and it is mostly interacted with using a Python
module named SLQAlchemy.

\section{Email Notification Subsystem}
\label{chap4:sec:email-sys}

\section{Firewall Handler Subsystem}
\label{chap4:sec:firewall-sys}

\section{Conclusions}
\label{chap4:sec:concs}
Cada capítulo \underline{intermédio} deve referir o que demais importante se conclui desta parte do trabalho, de modo a fornecer a motivação para o capítulo ou passos seguintes.
