\chapter{Technologies and Tools Used}
% OU \chapter{Trabalhos Relacionados}
% OU \chapter{Engenharia de Software}
% OU \chapter{Tecnologias e Ferramentas Utilizadas}
\label{chap:tech}

% we could argue here the case for linux android and the raspberry pi

\section{Introduction}
\label{chap3:sec:intro}
This project will make use of the Netfilter subsystem of the Linux kernel, this
module will be used for the capturing of packets for further analysis. As for
the mobile app it will be implemented for the Android operating system and it
will make use of the Java programming language.

\section{GNU/Linux}
\label{chap3:sec:gnux}
Gnu

\section{Raspberry Pi}
\label{chap3:sec:rasp}

\section{Packet Filtering System}
\label{chap3:sec:pfs}
The Packet Filtering System will be on the Raspberry Pi and will be responsible
for the capturing of all incoming packets to the system. It will make use of
the Netfilter susbsytem present in the Linux kernel, which willmake it possible
to implement hooks that will interact with the networking card of the Raspberry
Pi and it will enable the capture of every packet. This is one of the vital
parts of the system because of the efficiency gained from interacting with the
operating system's kernel directly.

By having access to every packet that arrives to the network card we can then
process them and rule according to the specified config file. For the filtering
per se we will use just plain C with arrays/lists  for the classification of
the protocols used.

\section{Android Companion App}
\label{chap3:sec:aca}
The system will communicate securely with the a Android application, this
process will use communications through SSL/TLS and it will communicate with
the filtering system, which runs on the Raspberry Pi.

The application itself will act as a Command\&Control and it will abstract the
user from all the work being done within the Packet Filtering System.

This companion application will connect to the server-side and enable the
network administrator to categorize any protocol as unsafe and to choose which
action it should perform once a connection from a device is made to the
Internet. The user could also monitor the traffic and access some metrics from
the activity of the Local Area Network devices. These metrics could be simple
percentages of unsafe connections or even refering to a specific protocol. It
could also log how many hours and the amount of traffic to a specific host, it
could also list all of the hosts visited from the devices inside the domestic
network.

% A tabela~\ref{tab:exemplo} serve apenas o propósito da exemplificação de como se fazem tabelas em \LaTeX.
% \begin{table}
% \centering
% \begin{tabular}{|c|lr|}
% \hline
% \textbf{campo 1} & \textbf{campo 2} & \textbf{campo 3}\\
% \hline
% \hline
% 14 & 15 & 16 \\
% \hline
% 13 & 13 & 13 \\
% \hline
% \end{tabular}
% \caption{Esta é uma tabela de exemplo.}
% \label{tab:exemplo}
% \end{table}

\section{Conclusions}
\label{chap3:sec:concs}
The focus of this chaper was on what technologies and components involved in
this project.
The use of low-level systems will help efficiency wise. This will also help by
making the analysis process as soon as the packets arrive to the machine.
% Cada capítulo \underline{intermédio} deve referir o que demais importante se conclui desta parte do trabalho, de modo a fornecer a motivação para o capítulo ou passos seguintes.
