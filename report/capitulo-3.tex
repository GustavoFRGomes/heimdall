\chapter{System Architecture}
% Os titulos dados aos capítulos são meros exemplos. Cada relatório deve adequar-se ao projeto desenvolvido.
\label{chap:sys-arch}

\section{Introduction}
\label{chap3:sec:intro}
This chapter will present the architecture chosen for the development of this
system, as well as the tools used for the development of the various components,
this chapter is meant to jsut give a general idea of the tools used not the
specific uses.

\section{Brief Requirement Analysis}
\label{chap3:sec:reqs}

O trecho de código seguinte mostra a função \texttt{main()} e o seu funcionamento:
\begin{lstlisting}[caption=Trecho de código usado no projeto.]
#include <stdio.h>

int main(){
  int i = 0;
  for(i = 0; i < 100; i++)
    printf("%d\n",i);
}
\end{lstlisting}


Se quiser definir a distribuição de Pareto, posso colocar a fórmula \emph{inline}, da seguinte forma $P(x)=\frac{x^{1/\Lambda}_{i}}{2}$, ou numa linha em separada, como se mostra a seguir:
$$ y^2 = \sum_{x=0}^{20}( x^3 - 2x + 3).$$

Outra maneira, mas numerada, é usar o ambiente \texttt{equation}, como se mostra na (\ref{eq:eq1}):
\begin{equation}
 y^2 = \sum_{x=0}^{20}( x^3 - 2x + 3).
 \label{eq:eq1}
\end{equation}

\begin{align}
 2+2+2+2+2+2+2+2+2+2+y^2 = & \sum_{x=0}^{20}( x^3 - 2x + 3);\\
                         = & x^4 -2.
 \label{eq:eq2}
\end{align}

\section{Architecture}
\label{chap3:sec:arch}
% Diagrama de Componentes
The system itself is all on the Raspberry Pi and the Android application is only
the interface that will enable the end-user interact with the firewall and also
receive notifications once the user is connected inside the network.

The components present on the Raspberry Pi are as follows:
\begin{enumerate}
	\item SQLite Database
	\item Web Service API
	\item Firewall Handler
	\item Email Notifier
\end{enumerate}

In the next chapter we will go into more details on these system and how they
all interact with each other. For now we can say that the Database play the
center role because everything needs to interact with it, either to get new
notifications, register flagged occurences or to interact with the web service.
% say database-centric?

\section{Tools}
\label{chap3:sec:tools}
In order to make the entire system feasable in the proposed timeframe there was
the need to use already available tools both in the Linux Operating System and
also on the Android

\subsection{Volley}
\label{chap3:sec:tools:sub:volley}
The Volley package is present in the standard library of Android it is
compatible with the first Android versions of the Operating System, and it helps
the developer to implement a queue of HTTP requests for various URL defined
resources.

This is a great tool for the development of web service consumers, making the
process of sending and receiving requests and responses with both text strings
or with JSON content. This package also helps the developer handle error
messages that might be received as a response from the web service.

\subsection{Flask}
\label{chap3:sec:tools:sub:flask}
The web service component for that runs on the Raspberry Pi was implemented
using the Flask framework. Flask is a Python micro-framework used for the development
of web applications but due to the freedom given the this framework extensions
could be created to make authentication, URL management, database interaction
and also web service development possible. In order to develop these solutions
the Flask framework makes use of extensions.

One of these extensions is called Flask-RESTful which enables developers to
implement RESTful APIs by declaring the Resources and the URIs that should be
available to the client. There are six principles that are considered teh
foundation of the REST architeture:
%Roy Fielding's Ph.D. dissertation that introduces the REST architecture
\begin{enumerate}
	\item Client-Server architecture with a clear separation between the two.
	\item Stateless -- every request from the client must contain all the
		pertinent information, which includes authentication credentials, for
		example. The server should not store any session information about the
		client.
	\item Cache -- responses from the client might be chacheable or not by the
		server or intermidiate systems in order to maximing performance. These
		responses must be labeled by the server as cacheable or noncacheable.
	\item Uniform Interface -- the protocol used between the client and the
		server resources must be well defined and standardized, for exmaple the
		use of the HTTP protocol.
	\item Layered Interface -- other machines and systems might be added between
		the client and the server for reliability, scalability or performance
		reasons.
	\item Code-on-Demand -- optionally, clients can dowload and execute code
		from the server in their own context.
\end{enumerate}

\subsection{SQLite}
\label{chap3:sec:tools:sub:sqlite}
The database handling is through the use of the another Python module named
SQLAlchemy which abstracts the developer from the SQL query language by using
Python classes to model the SQLite 3 database entities, and also query methods
for each class that allow the programmer to query and filter the results within
the database through Python methods.

The SQLite was the chosen RDBM due to it's low computational resource
requirements. This databae has a cap of data at 3TB which is more then suficient
for the purposes of this project.

\subsection{iptables}
\label{chap3:sec:tools:sub:iptables}
The iptables is known as the default firewall system deployed in almost all
GUN/Linux distributions. This firewall system enables the declaration of chains
for complex chains of events that can go from log appending as well as accepting
or blocking packets.

The iptables works by interacting withe the packet filtering hooks provided by
the Netfilter framework, which is present within the GNU/Linux kernel.
This firewall system also allows the interception of packets in various stages
of their traversal through the network card, due to the hooks from the Netfilter
framework. These built-in chains and their
respective hooks are as follows:
\begin{enumerate}
	\item PREROUTING -- NF_IN_PRE_ROUTING
	\item INPUT -- NF_IN_LOCAL_IN
	\item FORWARD -- NF_IP_FORWARD
	\item OUPUT -- NF_IP_LOCAL_OUT
	\item POSTROUTING -- NF_IP_POST_ROUTING
\end{enumerate}

\section{Conclusions}
\label{chap3:sec:concs}
In this chapter we have talked about the underlying architecture of the system
as well as the tools that were used in the various components of the project.
We refer the Volley used by the mobile application to request resources from the
Flask web service running a REST API. We also talked about the databate
management system as well as the SQLAlchemy Python module. We ended this chapter
by talking about the iptables firewall system present in GNU/Linux which will be
used as the firewall on the Raspberry Pi gateway.
