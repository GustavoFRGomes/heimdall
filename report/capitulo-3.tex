\chapter{System Architecture and Requirement Analysis}
% Os titulos dados aos capítulos são meros exemplos. Cada relatório deve adequar-se ao projeto desenvolvido.
\label{chap:sys-arch}

\section{Introduction}
\label{chap3:sec:intro}
This chapter focuses on software engineering with section 3.2 being a brief
requirement analysis for the system, followed a section describing the approach
to the system architecture, and then section 3.4 will focus on introducing the
tools that were leveraged in the implementation.

\section{Brief Requirement Analysis}
\label{chap3:sec:reqs}

This section will list both the function and non-functional requirements for the
system.

\subsection{Functional Requirements}
\label{chap3:sec:reqs:func}
The functional requirements for the system are comprised within the following
list:
\begin{enumerate}
	\item The system shall allow the user to add rules to the gateway firewall;
	\item The system shall allow the user to delete rules in the gateway
		firewall;
	\item The system shall allow the user to edit rules in the gateway's
		firewall;
	\item The system shall allow the user to list all the rules inside the
		gateway's firewall;
	\item The system shall allow the user to authenticate with the gateway
		application;
	\item The system shall allow the user to access the information of a rule in
		the gateway's firewall;
	\item The system shall allow the user to access all the packets that match a
		rule in the gateway's firewall;
	\item The system shall allow the user to access all the bytes that match a
		rule in the gateway's firewall;
	\item The system shall allow the user to add credentials on the gateway;
	\item The system shall allow the user to delete all user credentials stored
		in the gateway;
	\item The system shall allow the user to get the amount of packets for a
		specific rule;
	\item The system shall allow the user to get the amount of bytes for a
		specific rule;
	\item The system shall allow the user to add a notifications email;
	\item The system shall send traffic reports to the user's email;
\end{enumerate}

\subsection{Non-functional Requirements}
\label{chap3:sec:reqs:non-func}
\begin{enumerate}
	\item The system shall be divided using a Client-Server architetcure;
	\item The application shall only generate one database in the server-side;
	\item The application shall only allow for user registration on the
		server-side;
	\item The server-side component shall be divided into independent
		compontents;
	\item The server-side system shall comprise a RESTful web service;
	\item The server-side system shall comprise a Firewall service;
	\item The server-side system shall have an Email service to notify the user;
\end{enumerate}

\section{Architecture}
\label{chap3:sec:arch}
% Diagrama de Componentes
For this system there are two types of architecture being used, one is the
typical client-server architecture because it has the Android Companion
Application as the client and the system with which the application will
interact running on the Raspberry Pi, which is the server-side of the whole
system.

The server-side architecture was inspired by the
Microservices architecture being adopted in the Web Development are, which state
that a system should be divided into services that are independent from each
other in terms of execution and that each one of these services should only do
one single task or deal with one single component. This is reflected in the way
the server-side system was built being "database centric", i.e. each component
will interact solely with the database and not with each other preventing
concurrency issues upfront. The Raspberry Pi will then have the following
subsystems:
\begin{enumerate}
	\item Database;
	\item Web Service;
	\item Email Notifaction;
	\item Firewall Interface.
\end{enumerate}

This architecture design also enables the interchangeability of the server-side
components actually making it possible to implement alternatives for each of the
components and making them run without comprimising any of the other subsystem,
which makes it possible for the user or another developer to swap, for example,
the Email Notification with a \emph{\ac{SMS}} notification system
or the Firewall Interface with a new system that he or she implemented.

These distinct subsystems will be described in the next chapter, alongside the
description of the Android application.

% \subsection{Web Service}
% \label{chao3:sec:arch:web_service}

\section{Tools}
\label{chap3:sec:tools}
In order to make the entire system feasable in the proposed timeframe there was
the need to use already available tools both in the GNU/Linux Operating System
and also on the Android mobile \emph{\ac{OS}}.

\subsection{Volley}
\label{chap3:sec:tools:sub:volley}
Volley is a Java package devoloped by a team at Google Inc. with the intention
of creating asynchronous requests for network resources. It is focused on making
the efforts for \emph{\ac{HTTP}} requests much more feasable
within the Android Operation System. Since Android recommends that developers
avoid overloading the main thread of execution this packageprovides a simple
interface that enables the developers to simply declare a RequestQueue, which is
carried out on a separate thread, to handle the requests asynchronously.

In the context of this project the communication between the system running on
the Raspberry Pi and the mobile application is assured through the Web Service,
the Volley package was fulcral for the development of the mobile application
which relies heavily on requests from and to the network gateway, running on the
Raspberry Pi.

The Volley package also provides a easy way to handle JavaScrip Object Notation
(JSON) requests which is the format of the data supported by the Web Service.
The developers can declare a \emhp{\ac{HTTP}} request using whichever
\emhp{\ac{HTTP}} method is more suitables, i.e. GET, PUT, DELETE, POST.

The Volley package also allows the developer to create a different Network Stack
for each request making it possible to provide a version that deviates from the
standard solution. This was crucial for the development of the Android
application because of the use of the \emph{\ac{HTTPS}} protocol, enabling the
application to avoid any interception and manipulation of the data eing
transmitted to and from the gateway's web service, developed in this project.

It may also be noted that the headers for the authentication necessary for
accessing the web service were also comprised within the Volley requests which
made it posible for the Android application to support the \emph{\ac{HTTP}}
Basic Authentication headers.

\subsection{Flask}
\label{chap3:sec:tools:sub:flask}
The Flask framework is a small framework use mostly for the development of web
application. It's recent success is due to the fact that it is a very small
framework only providing the developer with four core components as is, i.e.
routing, debugging, a \emph{\ac{WSGI}} and Template Support.
Despite the lack of more functionality, the framework makes up for it by
supporting third-party extensions that allow for it's expansion in terms of
functionalities.

For this particular project that extensability was taken advantage of by using
the Flask as the basis for the Web Service being developed for the interaction
between the gateway and the user itself. The following list will consist of the
three extensions to the Flask framework which were used in the context of this
project:
\begin{enumerate}
	\item \emph{flask-sqlalchemy} -- which provides an abstraction between the
		SQL query language and the business logic.
	\item \emph{flask-restful} -- is the extension that enables the declaration
		of classes which will act as endpoints for each of the different Uniform
		Resource Identifiers (URIs).
	\item \emph{flask-httpauth} -- provides a seamless way to make all of the
		request authentication necessary.
\end{enumerate}


% \subsection{SQLite}
% \label{chap3:sec:tools:sub:sqlite}
% The database handling is through the use of the another Python module named
% SQLAlchemy which abstracts the developer from the SQL query language by using
% Python classes to model the SQLite 3 database entities, and also query methods
% for each class that allow the programmer to query and filter the results within
% the database through Python methods.

% The SQLite was the chosen RDBM due to it's low computational resource
% requirements. This databae has a cap of data at 3TB which is more then suficient
% for the purposes of this project.

\subsection{SQLAlchemy}
\label{chap3:sec:tools:sub:sqlalcemy}
SQLAlchemy is a library used to interact with the different SQL databases
available. This modules allows the conversion of the data within the database to
Python objects representing the models for each table. This library is a third
party module, not included in the Python Standard Libraries, that was created by
Mike Bayer in 2005.

This library is widely adopted within the Python developer universe because it
simplifies the access and manipulation of data inside a database. It supports
some of the most common database systems as, for example:
\begin{enumerate}
	\item Postgres
	\item MySQL
	\item SQLite
	\item Oracle
\end{enumerate}
By using this library a developer could abstract the code from the underlying
database and its \emph{\ac{SQL}} peculiarities. It also helps with the
sanitization of the data which is inputed to the database preventing common
security flaws like \emph{\ac{SQL}} injection attacks.

SQLAlchemy also provides flexibility by having two alternative modes of usage,
which could be used interchangeably depending on the personnal preferences of
the developer, these two distinct ways are:
\begin{enumerate}
	\item SQL Expression Language also known as Core -- is mostly a Pythonic way
		of representing common SQL statements and expressions with almost no
		abstraction from typical SQL language. This is also the foundation of
		the next usage mode.
		% ORM - Object Relational Mapper
	\item ORM -- which stands for Object Relational Mapper, has a more
		high-level abstraction and leveraging a declarative system that enables
		the developer to work in a more idiomatic way. This is accomplished by
		binding the database schema with data models represented as Python
		classes.
\end{enumerate}

This project makes extensive use of the SQLAlchemy's ORM mode with a SQLite3
database system.

\subsection{iptables}
\label{chap3:sec:tools:sub:iptables}
The iptables is known as the default firewall system deployed in almost all
GUN/Linux distributions. This firewall system enables the declaration of chains
for complex chains of events that can go from log appending as well as accepting
or blocking packets.

The iptables works by interacting withe the packet filtering hooks provided by
the Netfilter framework, which is present within the GNU/Linux kernel.
This firewall system also allows the interception of packets in various stages
of their traversal through the network card, due to the hooks from the Netfilter
framework. These built-in chains and their
respective hooks are as follows:
\begin{enumerate}
	\item PREROUTING -- NF\_IN\_PRE\_ROUTING
	\item INPUT -- NF\_IN\_LOCAL\_IN
	\item FORWARD -- NF\_IP\_FORWARD
	\item OUPUT -- NF\_IP\_LOCAL\_OUT
	\item POSTROUTING -- NF\_IP\_POST\_ROUTING
\end{enumerate}

In order to interact with iptables on the Raspberry Pi, the system makes use of
\emph{python-iptables} which is an extension module that makes a brigde between
iptables and the Python programming language.
This module makes use of Python objects to perform all the functionality for the
firewall, including interacting and defining:
\begin{enumerate}
	\item Chains;
	\item Tables;
	\item Rules;
\end{enumerate}

The iptables application has the following four tables: (1) FILTER; (2) MANGLE;
(3) RAW; (4) not sure. These tables allow for different purposes, and for the
scope of this project only the \emph{FILTER} table is important, since the
primary objective of the system being developed is to filter packets and
connections according to rules established by the user.

Python-iptables, also known as \emph{iptc}, allows the deveolper to access and
interact with the firewall system for any of the aforementioned Tables, and
allows the creation and management of the Chains within these tables.

The iptables Chains are sets of rules that can be invoked whenever a packet
matches the rules within that chain, iptables also provides three different
predefined Chains which are:
\begin{enumerate}
	\item \emph{INPUT} -- for all packages that have as host the machine to
		which they arrive;
	\item \emph{FORWARD} -- covers all the network traffic that isn't destinted
		nor originated from the machine in question;
	\item \emph{OUTPUT} -- is the chain that handles all the traffic generated
		by the host.
\end{enumerate}

Since the main functionality of this project is to filter traffic that is
passing through the gateway, the main Table and Chain that will be used are
FILTER and FORWARD, respectively.

\section{Conclusions}
\label{chap3:sec:concs}
In this chapter we have talked about the underlying architecture of the system
as well as the tools that were used in the various components of the project.
We refer the Volley used by the mobile application to request resources from the
Flask web service running the \emph{\ac{API}}. We also talked about the databate
management system as well as the SQLAlchemy Python module. We ended this chapter
by talking about the iptables firewall system present in GNU/Linux which will be
used as the firewall on the Raspberry Pi gateway.
