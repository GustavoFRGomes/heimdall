\chapter{Prototype and Testing}
\label{chap:proto-test}

\section{Introduction}
\label{chap4:sec:intro}
Cada capítulo \underline{intermédio} deve começar com uma breve introdução onde é explicado com um pouco mais de detalhe qual é o tema deste capítulo, e como é que se encontra organizado (i.e., o que é que cada secção seguinte discute).
This chapter will describe the Prototype developed for this project as well as
the Environment in which the system was tested. It will also talk about the
process of installation, usage. Only a protoype of the system was made, even
though it is a complete system there are some feature that could make the system
more appealing to every user.

\section{Prototype Description and Environment}
\label{chap4:sec:desc-env}
The system can be deployed using the initializing script, which will kick off
all systems involve. Since we have multiple systems there is the need to start
every system seperately. This will be done by running the
\emph{heimdall\_launcher.py} script and typing all of the information that the
initiliazation script, which is inside the \emph{heimdall\_init.py} source file,
will ask. The information is mostly the user's name, email contact information
and to add a user to the database in order to perform authentication.

This section is divided in the following subsections:
\begin{enumerate}
	\item Raspberry Pi Preparation
	\item Server-Side Dependencies
	\item Android Application
\end{enumerate}

\subsection{Raspberry Pi Preparation}
\label{chap5:sec:inst-usg:sub:rasp-prep}
For the Raspberry Pi to behave as the inteded gateway for the network it needs
to at least be the machine that attributes the local IP addresses and declare
itself as the domestic network gateway. For this the Raspberry Pi needs to run
the \emph{dhcp} server.

Another vital part of the system is the presence of the Python interpreter,
especially the version 3.4.x. This is needed in order to run the system itself,
excluding the Android application. Another resource that is encoraged to have is
the \emph{pip} application which is a package manager for Python packages, the
required packages are described in the next subsection.

\subsection{Server-Side Dependencies}
\label{chap5:sec:inst-usg:sub:rasp-prep}
There are a few dependencies that should be downloaded and installed in order to
run all of the components of the system.

\begin{enumerate}
	\item sqlalchemy -- Database Interface
	\item flask -- Micro-framework for the web service
	\item flask-restful -- Extension for Flask to create the RESTful API within
		the Web Service.
	\item flask-sqlalchemy -- Extension for Flask to integrate SQLAlchemy with
		the Web Service.
	\item flask-httpauth -- Extension responsible for the integration and
		management of the login, which is required by the web service.
	\item python-iptables -- Interface Module for the iptables management and
		manipulation.
\end{enumerate}

In order to download and install these Python dependencies the user could use
\emph{pip} and issue the following comman

\section{Testing}
\label{chap4:sec:testing}
In order to perform the teting of the system a couple of helper scripts were
developed, these can be found inside the \emph{testing} folder on the project's
root directory.

\subsection{Web API Tester}
\label{chap5:sec:testing:api}
A testing script was developed in order to test all of the functionality that
the web service must provide, since communication is perform through the Hyper
Test Tranfer Protocol (HTTP) and the data is in the JavaScript Object Notation
(JSON) it is possible to make simple HTTP requests as said in previous
chapters.

This script is named \emph{web\_api.py} and it can be imported into the Python
interpreter, by running the Python interpreter inside the directory and simply
typing the import command. This will make available some variables stored within
it, for example a list of Uniform Resource Identifiers (URIs) that are available
for access.
Since the web service is only available through HTTPS and requires user
credentials to be inserted with each request, the testing file also contains the
following global variables:
\begin{enumerate}
	\item USERNAME -- which stores the user's username, one that is registered
		inside the database or the authentication will fail.
	\item PASSWORD -- storing the user's password which needs to match the value
		stored for the user's username.
	\item VERIFY -- is mainly a flag for the HTTPS, in order to make it possible
		to commnicate through HTTPS and not plain HTTP.
\end{enumerate}


The function that can test range from the types of requests that need to be
made, mostly GET, POST, PUT and DELETE types of requests. There are functions to
handle the submission of Rule into the system as well as the deletion and
listing of all of them.

All of these functions have some documentation to test all of the API requests
that are available.

\subsection{Firewall Tests}
\label{chap5:sec:testing:firewall}
The Firewall is tested manually and it only requires the rules to be inside the
database and then the user may only run the \emph{firewall.py} as a Python
script by appending the source file name to the \emph{python} command in the
terminal enabling it to self-execute with a cycle of 30 seconds as the default
value on the FORWARD chain of the iptables.
If there are already rules inside the database they will be converted and
submitted to the \emph{iptables} system but there can be added rules through the
database that will be updated in the next cycle of the firewall.

\section{Conclusions}
\label{chap4:sec:concs}
Cada capítulo \underline{intermédio} deve referir o que demais importante se conclui desta parte do trabalho, de modo a fornecer a motivação para o capítulo ou passos seguintes.
