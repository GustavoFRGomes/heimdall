\chapter{Prototype and Testing}
\label{chap:proto-test}

\section{Introduction}
\label{chap4:sec:intro}
Cada capítulo \underline{intermédio} deve começar com uma breve introdução onde é explicado com um pouco mais de detalhe qual é o tema deste capítulo, e como é que se encontra organizado (i.e., o que é que cada secção seguinte discute).
This chapter will describe the Prototype developed for this project as well as
the Environment in which the system was tested. It will also talk about the
process of installation, usage. Only a protoype of the system was made, even
though it is a complete system there are some feature that could make the system
more appealing to every user.

\section{Prototype Description and Environment}
\label{chap4:sec:desc-env}
The system can be deployed using the initializing script, which will kick off
all systems involve. Since we have multiple systems there is the need to start
every system seperately. This will be done by running the \emph{heimdall.py}
script.


\section{Installation and Usage}
\label{chap4:sec:inst-usg}
In this section we are going to talk about the dowload and installation of the
deoendencies that the system requires as well as the configuration that needs to
be done by the user to make the system run as it is supposed to do.

This section is divided in the following subsections:
\begin{enumerate}
	\item Raspberry Pi Preparation
	\item Server-Side Dependencies
	\item Android Application
\end{enumerate}

\subsection{Raspberry Pi Preparation}
\label{chap5:sec:inst-usg:sub:rasp-prep}
For the Raspberry Pi to behave as the inteded gateway for the network it needs
to at least be the machine that attributes the local IP addresses and declare
itself as the domestic network gateway. For this the Raspberry Pi needs to run
the \emph{dhcp} server.

Another vital part of the system is the presence of the Python interpreter,
especially the version 3.4.x. This is needed in order to run the system itself,
excluding the Android application. Another resource that is encoraged to have is
the \emph{pip} application which is a package manager for Python packages, the
required packages are described in the next subsection.

\subsection{Server-Side Dependencies}
\label{chap5:sec:inst-usg:sub:rasp-prep}
There are a few dependencies that should be downloaded and installed in order to
run all of the components of the system.

\begin{enumerate}
	\item sqlalchemy
	\item flask
	\item flask-restful
	\item flask-sqlalchemy
	\item python-iptables
\end{enumerate}

In order to download and install these Python dependencies the user could use
\emph{pip} and issue the following comman

\section{Testing}
\label{chap4:sec:testing}

\section{Conclusions}
\label{chap4:sec:concs}
Cada capítulo \underline{intermédio} deve referir o que demais importante se conclui desta parte do trabalho, de modo a fornecer a motivação para o capítulo ou passos seguintes.
