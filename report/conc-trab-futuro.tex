\chapter{Conclusions and Future Work}
\label{chap:conc-future-work}

\section{Main Conclusions}
\label{sec:main-concs}

% Esta secção contém a resposta à questão: \\
% \emph{Quais foram as conclusões princípais a que o(a) aluno(a) chegou no fim deste trabalho?}
This project consisted inthe development of a software slution that allows the
final user to configure and manage the firewall rules on the firewall of the
domestic network. This is achieved with teh use of a Raspberry Pi running a set
of services that allow for the configuration of the firewall rules as well as
the communication with interfacing devices with it's web service and also the
notification center which sends out email reports of the traffic within the
user's domestic network.

Altough the system is inteded to be easy to use even for layman, there is still
some improvements to be made, especially because the user must know which IP
and/or MAC address is allocated to the specific device the user wants to block
or flag. This will be discussed in the next section of this chapter.

Another functionality that couldn't be implemented was the conversion of
unsecure connections to anonymous connections through the Tor network but
unfortunately there was no time and especially it was something that needed the
Raspberry Pi to become much more similar to a network router instead of a simple
gateway because it would need to anonymize and maintain the connection status
for each and every connection from within the domestic network and the Internet.
This topic will also be discussed in the section 6.2.

\section{Future Work}
\label{sec:future-work}

Unfortunately the anonymization through the Tor network couldn't be accomplished
mostly because it would need a redesign of the system and more control over the
network's trafic. It is planned to be a feature of the next iteration of the
project, mostly because it is important to give the user a way of preventing
interception of unsecure traffic from devices like the smart "things" that will
comprise the Internet of Things (IoT).

Another aspect that needs some revision and a redesign is the Android
application, even though the system doesn't require the exclusive use of the
Android application, most users could still make use of it. This makes it vital
to have a User Interface more intuitive and easy to use even for the
non-technical user.

One more problem that can be solved with more time is the identification and
mapping of the devices connected to the domestic network. This would make it
possible to simply list these devices to the user and he or she would only need
to select the one they wanted to monitor more closely. This is currently done by
allowing the user to specify the Internet Protocol (IP) and/or Machine
Authentication Code (MAC) addresses which isn't suitable for every user because
they would need to identify by themselves these addresses and then add the rule.
The development of the identification and monitoring system needs to integrate
the solution in order to abstract all the technical skill required to operate
the system as it is.

Another improvement that isn't vital but could also be a good feature to have
in the future, would be the rule interface for IPv6 addresses. This wasn't
already accomplished due to time constraints and also because of the Dynamic
Host Configuration Protocol (DHCP) configuration is limited to the attribution
of IPv4 addresses to the devices that connect to the network.

% Expansion into full firewall capabilities for incoming traffic as well
% The main desire of the system could be expanded to work as a firewall for both
% incoming and outgoing communications instead of being just for the outgoing
% traffic. This would increase security even though the domestic router also
% handles the firewall duties for incoming traffic, but this system could help
% with further security and filtering ouside from not letting any incoming traffic
% that wasn't requested by a device inside the Local Area Network.

% This system could also have an Intrusion Detection System for both the incoming
% and outgoing traffic but also for the inner traffic, this would prevent the
% infection of devices inside the Local Area Network minimazing the risk of worm
% spreading all across the entire domestic network. The IDS system could also
% enable to identify possible "zombie" devices inside the network by analyzing
% patterns in the traffic and notifying the user for those anomalies.

% One more feature that will be developed is the implementation of a firewall
% software, instead of using a third party system, with the intention of making
% the system more efficient for the Raspberry Pi which is not very powerful both
% in terms of memory and processing power. It would also be interesting to test
% the system in the newer Raspberry Pi 3 which as a better processor and compare
% the results.
% Talk about making an IDS system for intranetwork malware detection, it could
% be a way to stop malware from spreading to some of the connected devices, talk
% about detection of information extraction system, implementing our own
% firewall towards having it more optimized for the Raspberry Pi.

% Coming up with ways to identify every device that is connected, and blocking
% access to a particular host on the Internet as well as any communications
% coming from IoT devices for some reason, having DNS lookup before.
% Another feature that is pretended to be implemented in the future is a system
% for the detection of devices to aid in the detection and identification of
% possible unkown devices that are connected to the domestic network. It could
% also help identify all the IoT devices in the network to compose individual
% reports on those same devices.

% The email notifications could be done using an outside server ran by me and
% the connection could be made from the Raspberry Pi into the server for email
% notifications on the fly.
% Lastly the email notification system could be done using an outside service, one
% which could be accessed in order to send the emails to the user from an
% authentic email server without needing to provide credentials for a known STMP
% server as it is done currently.

% Esta secção responde a questões como:\\
% \emph{O que é que ficou por fazer, e porque?}\\
% \emph{O que é que seria interessante fazer, mas não foi feito por não ser exatamente o objetivo deste trabalho?}\\
% \emph{Em que outros casos ou situações ou cenários -- que não foram estudados no contexto deste projeto por não ser seu objetivo -- é que o trabalho aqui descrito pode ter aplicações interessantes e porque?}
