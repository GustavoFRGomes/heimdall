\chapter{Conclusions and Future Work}
\label{chap:conc-future-work}

\section{Main Conclusions}
\label{sec:main-concs}

% Esta secção contém a resposta à questão: \\
% \emph{Quais foram as conclusões princípais a que o(a) aluno(a) chegou no fim deste trabalho?}
This project consisted on the development of a software solution that enables
the user to configure and manage the rules within the database. This was
accomplished by having a Raspberry Pi running a set of different services that
update the firewall rules, perform notifications through email and also accept
requests to interact with the system.

Altough the system is inteded to be easy to use even for layman, there is still
some improvements to be made, especially because the user must know which IP
and/or MAC address is allocated to the specific device the user wants to block
or flag. This will be discussed in the next section of this chapter.

Another functionality that couldn't be implemented was the conversion of
unsecure connections to anonymous connections through the Tor network but
unfortunately there was no time and especially it was something that needed the
Raspberry Pi to become much more similar to a network router instead of a simple
gateway because it would need to anonymize and maintain the connection status
for each and every connection from within the domestic network and the Internet.
This topic will also be discussed in the section 6.1.

\section{Future Work}
\label{sec:future-work}

Altough just one goal from the start of this project wasn't achieved there could
also be some implementation improvements and some optimizations and a redesign
of the Android companion application.

First and foremost the Tor anonymization of unsecure connections needs to be
made in the next iteration but maybe as a solo project that could work outside
the scope of the system described in this document. This would help improve the
overall security especially in terms of the interception of traffic by malicious
agents.

Another improvement much needed in this system is the Android application which
lacks a intuitive User Interface (UI) that will enable even the non-technical
users to work and interact with the system.

We also can't expect the user to
know and have working knowledge about Internet Protocol (IP) and
Machine Authentication Code (MAC) addresses so there needs to also be a
mechanism of identification for each device connected to the domestic network.
This is actually on plan for the next iteration of the project because the aim
is to make it a solution accessible to anyone, in terms of technicall skills
required.

Something that could also be done and make the system itself more valuable would
be the filtering and detection of IPv6 rules, for now the database already has
an attribute that represents if the IP is IPv4 ou IPv6 and also the validation
for IPv6 addresses is also already implemented. The main reason why this isn't
implemented is that the domestic network will be configured with the \emph{dhcp}
application and a IPv4 private network and IP range is the configured instead of
letting the user decide between that and the newer IPv6 range.

% Expansion into full firewall capabilities for incoming traffic as well
% The main desire of the system could be expanded to work as a firewall for both
% incoming and outgoing communications instead of being just for the outgoing
% traffic. This would increase security even though the domestic router also
% handles the firewall duties for incoming traffic, but this system could help
% with further security and filtering ouside from not letting any incoming traffic
% that wasn't requested by a device inside the Local Area Network.

% This system could also have an Intrusion Detection System for both the incoming
% and outgoing traffic but also for the inner traffic, this would prevent the
% infection of devices inside the Local Area Network minimazing the risk of worm
% spreading all across the entire domestic network. The IDS system could also
% enable to identify possible "zombie" devices inside the network by analyzing
% patterns in the traffic and notifying the user for those anomalies.

% One more feature that will be developed is the implementation of a firewall
% software, instead of using a third party system, with the intention of making
% the system more efficient for the Raspberry Pi which is not very powerful both
% in terms of memory and processing power. It would also be interesting to test
% the system in the newer Raspberry Pi 3 which as a better processor and compare
% the results.
% Talk about making an IDS system for intranetwork malware detection, it could
% be a way to stop malware from spreading to some of the connected devices, talk
% about detection of information extraction system, implementing our own
% firewall towards having it more optimized for the Raspberry Pi.

% Coming up with ways to identify every device that is connected, and blocking
% access to a particular host on the Internet as well as any communications
% coming from IoT devices for some reason, having DNS lookup before.
% Another feature that is pretended to be implemented in the future is a system
% for the detection of devices to aid in the detection and identification of
% possible unkown devices that are connected to the domestic network. It could
% also help identify all the IoT devices in the network to compose individual
% reports on those same devices.

% The email notifications could be done using an outside server ran by me and
% the connection could be made from the Raspberry Pi into the server for email
% notifications on the fly.
% Lastly the email notification system could be done using an outside service, one
% which could be accessed in order to send the emails to the user from an
% authentic email server without needing to provide credentials for a known STMP
% server as it is done currently.

% Esta secção responde a questões como:\\
% \emph{O que é que ficou por fazer, e porque?}\\
% \emph{O que é que seria interessante fazer, mas não foi feito por não ser exatamente o objetivo deste trabalho?}\\
% \emph{Em que outros casos ou situações ou cenários -- que não foram estudados no contexto deste projeto por não ser seu objetivo -- é que o trabalho aqui descrito pode ter aplicações interessantes e porque?}
