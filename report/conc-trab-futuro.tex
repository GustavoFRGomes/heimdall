\chapter{Conclusions and Future Work}
\label{chap:conc-future-work}

\section{Main Conclusions}
\label{sec:main-concs}

% Esta secção contém a resposta à questão: \\
% \emph{Quais foram as conclusões princípais a que o(a) aluno(a) chegou no fim deste trabalho?}

\section{Future Work}
\label{sec:future-work}

% Expansion into full firewall capabilities for incoming traffic as well
The main desire of the system could be expanded to work as a firewall for both
incoming and outgoing communications instead of being just for the outgoing
traffic. This would increase security even though the domestic router also
handles the firewall duties for incoming traffic, but this system could help
with further security and filtering ouside from not letting any incoming traffic
that wasn't requested by a device inside the Local Area Network.

This system could also have an Intrusion Detection System for both the incoming
and outgoing traffic but also for the inner traffic, this would prevent the
infection of devices inside the Local Area Network minimazing the risk of worm
spreading all across the entire domestic network. The IDS system could also
enable to identify possible "zombie" devices inside the network by analyzing
patterns in the traffic and notifying the user for those anomalies.

One more feature that will be developed is the implementation of a firewall
software, instead of using a third party system, with the intention of making
the system more efficient for the Raspberry Pi which is not very powerful both
in terms of memory and processing power. It would also be interesting to test
the system in the newer Raspberry Pi 3 which as a better processor and compare
the results.
% Talk about making an IDS system for intranetwork malware detection, it could
% be a way to stop malware from spreading to some of the connected devices, talk
% about detection of information extraction system, implementing our own
% firewall towards having it more optimized for the Raspberry Pi.

% Coming up with ways to identify every device that is connected, and blocking
% access to a particular host on the Internet as well as any communications
% coming from IoT devices for some reason, having DNS lookup before.
Another feature that is pretended to be implemented in the future is a system
for the detection of devices to aid in the detection and identification of
possible unkown devices that are connected to the domestic network. It could
also help identify all the IoT devices in the network to compose individual
reports on those same devices.

% The email notifications could be done using an outside server ran by me and
% the connection could be made from the Raspberry Pi into the server for email
% notifications on the fly.
Lastly the email notification system could be done using an outside service, one
which could be accessed in order to send the emails to the user from an
authentic email server without needing to provide credentials for a known STMP
server as it is done currently.

% Esta secção responde a questões como:\\
% \emph{O que é que ficou por fazer, e porque?}\\
% \emph{O que é que seria interessante fazer, mas não foi feito por não ser exatamente o objetivo deste trabalho?}\\
% \emph{Em que outros casos ou situações ou cenários -- que não foram estudados no contexto deste projeto por não ser seu objetivo -- é que o trabalho aqui descrito pode ter aplicações interessantes e porque?}
