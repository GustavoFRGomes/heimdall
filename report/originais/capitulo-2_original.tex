\chapter{State of the Art}
% OU \chapter{Trabalhos Relacionados}
% OU \chapter{Engenharia de Software}
% OU \chapter{Tecnologias e Ferramentas Utilizadas}
\label{chap:state-of-art}

\section{Introduction}
\label{chap2:sec:intro}
Nowadays there has been a number of cases that work reveal the insecurity of
most devices on the Internet of Things %unfinished.
% put some references here maybe!!
but how do regular people protect themselves from the perils their devices
might bring to their household? There isn't a solution to protect their own
network from anyone who could leverage their "smart" appliances to get inside
it and perform malicious actions, like traffic monitoring, sensitive
information extraction or even make your devices part of a botnet.

The focus of this project is on the traffic that willbe sent from devices onto
the Internet, the outgoing traffic because it represents a point of failure.
If the outgoing traffic isn't secure then anyone could study our daily life,
form the information sent from any "smart" appliances an attacker could study
a potential victim and spy on their lives. At this present time there are a
number of cameras that are connected to the Intenret, thus being part of the
Internet of Things, that are reachable by anyone on the Internet and that
could be viewed by anyone who as a minimum skill level for michieves.

\section{Wireshark}
\label{chap2:sec:wire}
% Describe the monitoring of the Internet traffic monitoring
Wireshark is a network monitoring tool that captures network packets from the
Local Area Network. This tool is mostly used by
\section{Nagios}
\label{chap2:sec:nag}
% Described Nagios which is similar to what we are trying to accomplish

One similar product to the one we are developing is Nagios, this is a network
infrastruture monitoring system designed to map an

Nagios is a infrastructure monitoring tool, developed mostly for commercial and
industrial uses. This is a tool that can add connected devices and monitor them
and all of the traffic they generate. Nagios has a version especially made for
the Raspberry Pi, called the NagiosPi, that consists of 5 different interfaces
the user could interact with.
% make a list here with the following
Nagios;
Nconf;
NagVis;
PHPMyAdmin;
RaspControl.

The NagiosPi system enables the monitoring of all traffic coming in and out of
the devices, logging it if intructed. It could monitor specific ports from
specific machines. It also has a visualization functionality which lets its
users visualize all the devices that are connected.

\section{Privacy Concerns}
\label{chap2:sec:priv}
% put some of the images from the first presentation of the IP cams.
There has been some worries about the privacy that some of the devices
connected to the Internet might have, e.g. IP cams. These devices are very
limited in terms of computatioinal resources so they don't implement some basic
security features and as a result anyone could access some of those devices
inside the domestic network from around the world.
There are search engines for devices connected to the Internet which can
identify some devices connected to the Internet and that are unsecure to a
point where anyone could access the information hoozing out to the Internet.

With these search engines anyone could find for example a remote surveillance
camera, or IP cam, which is reachable from the Internet and which could be used
to spy on people's lives. Privacy is a concept which has been talked about in
today's world and these cameras and systems are simply making any footage
accessible by anyone online, mostly this is not done by the manufacturers as a
way to let anyone spy on us but rather a negligence based on the lack of power
and resources those machines have, which make it harder if not impossible to
secure them with the computational resources at hand so they don't shield the
users privacy over functionality.

\section{The Internet of Things}
\label{chap2:sec:iot}
% talk a lot about the Internet Of Things
The Internet of Things is becoming a hugely adopted term to describe all of the
devices that are connected to the Internet, being smartphones, tablets, smart
fridges, coffee markers and most appliances that are now able to connect to the
Internet and communicate with other devices or even with people.

A humongous connected world is becoming more and more a reality instead of a
profecy/fantasy. Since the opening of the Internet to the general public we
have seen a exponential increase in the number of devices connected at any
given time. No one can argue that the Internet is not the biggest social
experiment ever accomplished, it unites and dissiminates knowledge among it's
users.

Companies see the value of selling connected devices and they show futuristic
predictions of how the world could look like if more and more devices were to
be connected to the Internet, they sell us the idea that our fridge might know
that we are out of milk and order some online without we even noticing, they
sell us on the idea that our lives would be easier if our devices could be
connected to the Internet and automate some household maintenance for us. But
they won't tell us about the threats that we could have if even one of those
devices is remotely compromised.

Most people are not aware or even technically skilled to know if and when their
Internet connected devices are using unsecure protocols for communications. The
networking knowledge and tools that are needed to identify which devices and
which protocols are being used on every device.

\section{Conclusions}
\label{chap2:sec:concs}
This chapter provides context to the problem at hand, by describing the world
that we live in right now and also the neglect that some companies are
manufacturing unsecure devices and consumers are unware of this when they are
buying their smart appliances.

This chapter also makes a brief description of what is the Internet of Things
and what are the devices that can be included in this network.
