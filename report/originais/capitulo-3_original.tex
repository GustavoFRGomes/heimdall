\chapter{Technologies and Tools Used}
% OU \chapter{Trabalhos Relacionados}
% OU \chapter{Engenharia de Software}
% OU \chapter{Tecnologias e Ferramentas Utilizadas}
\label{chap:tech}

% we could argue here the case for linux android and the raspberry pi

\section{Introduction}
\label{chap3:sec:intro}
This project will make use of the Netfilter subsystem of the Linux kernel, this
module will be used for the capturing of packets for further analysis. As for
the mobile app it will be implemented for the Android operating system and it
will make use of the Java programming language.

\section{Netfilter}
\label{chap3:sec:netfilter}
The Linux kernel has a framework that enables the interception and capture of
packets that are travelling through the network stack of any device that is
running Linux. The system is called Netfilter and permits the registration of
hooks at specific points in the network stack which allow an aplication to
manipulate and even to delete packets from the stack. This is a very powerful
framework that makes it possible to implement firewalls with less of a hassle.
The \emph{iptables} firewall system present in most Linux distribution uses this
system for its packet filtering capabilities, enabling the user of this
application to perform actions based on the hooks
% Talk about the fact that the hooks could be specified when filtering packets
% in the iptables POSTROUTING, PREROUTING, etc...

In the Linux kernel there is a framework capable of deploying hooks to the
network stack in order to manipuate the packets that are received. This
framework is used by \emph{iptables}, the firewall system that comes in all the
Linux distributions as default, to implement the rules of a firewall.

The hooks allow packets to be intercepted, while they are travelling through the
network stack of the device and these can be accessed in the following
instances:
% insert all of the hooks
\begin{enumerate}
	\item Pre-Routing
	\item Post-Routing
	\item Put them all here
\end{enumerate}
% Describe when one of those hooks occurs, what happens or where the packet is
% when the hooks is triggered

\section{GNU/Linux}
\label{chap3:sec:gnux}
GNU/Linux is the operating system environment that the system will run, it will
also make use of the Netfilter framework which is a component of the Linux
kernel. This operating system enables the use of less resources in comparison to
other Operating Systems available at the present time.
By using this OS the resources available for the application are sufficient even
for larger networks, another reason is the availability of low-level controls
ready for a developer to utilize, like the framework that enables the
implementation of the firewall capabilities of the project being developed.

Of course GNU/Linux is not the only operating system available with these
capabilities but it is the most adopted and also enables better performance then
most of its competitors.

\section{Raspberry Pi}
\label{chap3:sec:rasp}
The Raspberry Pi board is a small computer, in the sense that it has all of the
components of the von Neumann architecture, it has a Central Processing Unit
(CPU), memory and input and output slots for external devices.
For this project there was the use of a Raspberry Pi model 2, that boasts a
quad-core processor with a clock speed of 800MHz, one gigabyte of RAM (RANDOM
ACCESS MEMORY) and has a networking port and a microSD card slot. With these
features a Linux distribution is ran and it will act as a software firewall and
a default gateway giving it access to all the networks traffic, this way it can
fitler it according with the guidelines provided by the end-user of the system.

Choosing the Raspberry Pi as the device in which to run the system was simple.
Since this particular model and the new Raspberry Pi model 3 are available
between the price of 30 and 40€ it is an inexpensive machine that can function
as a gateway to outgoing traffic and anyone could buy one and install the system
in order to protect itself.

All of the system will be constrained by the lack of greater power but since the
main purpose of this device is to filter the outgoing traffic it will be
suficient even for bigger networks.

Tests should be done on both the Raspberry Pi 2 and 3 since they differ when it
comes to CPU performance. The main focus will be on the Raspberry Pi 2 but once
that is done a few tests should occur to check if it would be better performant
on the model 3.

\section{Packet Filtering System}
\label{chap3:sec:pfs}
The Packet Filtering System will be on the Raspberry Pi and will be responsible
for the capturing of all incoming packets to the system. It will make use of
the Netfilter susbsytem present in the Linux kernel, which willmake it possible
to implement hooks that will interact with the networking card of the Raspberry
Pi and it will enable the capture of every packet. This is one of the vital
parts of the system because of the efficiency gained from interacting with the
operating system's kernel directly.

By having access to every packet that arrives to the network card we can then
process them and rule according to the specified config file. For the filtering
per se we will use just plain C with arrays/lists  for the classification of
the protocols used.

\section{Android}
\label{chap3:sec:andrd}
Android  is an operating system developed for mobile platforms in particular for
smartphones and later also for tablets. The underlying technology was bough by
Google back in ......... and it has since evolved into a fully formed operating
system capabale of also working with limited resources present in mobile deices.
At it's heart it has a stripped down version of the Linux kernel, built to
provide good power to hardware performance.
% Talk about the wide use of the OS in mobile devices

By developing the companion application of this project in this operating system
two thing will be accomplished, there will be a bigger availability because of
the world wide adoption of this particular operating system and also it will
enable the end user to control the entire system with their mobile devices,
e.g., most people carry around their phone even when they are home so, by
providing a mobile application any user could access the system, and also
receive immediate notifications if something out of the ordinary is happening
inside their network.

This companion application will only work inside the network and it will
implement security mearsures in order to disable any connection from a possible
attacker. The only thing that could be risked is if the device that has the
companion application is compromised.

\section{Android Companion App}
\label{chap3:sec:aca}
The system will communicate securely with the a Android application, this
process will use communications through SSL/TLS and it will communicate with
the filtering system, which runs on the Raspberry Pi.

The application itself will act as a Command\&Control and it will abstract the
user from all the work being done within the Packet Filtering System.

This companion application will connect to the server-side and enable the
network administrator to categorize any protocol as unsafe and to choose which
action it should perform once a connection from a device is made to the
Internet. The user could also monitor the traffic and access some metrics from
the activity of the Local Area Network devices. These metrics could be simple
percentages of unsafe connections or even refering to a specific protocol. It
could also log how many hours and the amount of traffic to a specific host, it
could also list all of the hosts visited from the devices inside the domestic
network.

% A tabela~\ref{tab:exemplo} serve apenas o propósito da exemplificação de como se fazem tabelas em \LaTeX.
% \begin{table}
% \centering
% \begin{tabular}{|c|lr|}
% \hline
% \textbf{campo 1} & \textbf{campo 2} & \textbf{campo 3}\\
% \hline
% \hline
% 14 & 15 & 16 \\
% \hline
% 13 & 13 & 13 \\
% \hline
% \end{tabular}
% \caption{Esta é uma tabela de exemplo.}
% \label{tab:exemplo}
% \end{table}

\section{Conclusions}
\label{chap3:sec:concs}
The focus of this chaper was on what technologies and components involved in
this project.
The use of low-level systems will help efficiency wise. This will also help by
making the analysis process as soon as the packets arrive to the machine.
% Cada capítulo \underline{intermédio} deve referir o que demais importante se conclui desta parte do trabalho, de modo a fornecer a motivação para o capítulo ou passos seguintes.
